\structure{ЦЕЛЬ РАБОТЫ}

Изучить теоретико-игровую модель информационного противоборства в
социальных сетях. Найти аналитическое решение игры c не
противоположными интересами двух игроков и определить итоговые
расстояния до точки утопии.

\subsection*{Задание}

Необходимо проделать следующие шаги для выполнения работы:

\begin{itemize}
  \item Для 10 агентов случайным образом сгенерировать стохастическую матрицу доверия.
  \item Назначить всем агентам случайное начальное мнение из заданного
        отрезка числовой оси. Найти итоговое мнение агентов.
  \item Случайным образом выбрать количество и номера (непересекающиеся)
        агентов влияния из общего числа агентов для первого и второго игроков.
        Назначить им начальные мнения первого и второго игроков. Остальным
        агентам (нейтральным) назначить случайные начальные мнения.
        Смоделировать информационное управление в рамках игры и
        определить итоговое мнение агентов.
\end{itemize}
