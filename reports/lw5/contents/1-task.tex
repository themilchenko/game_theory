\structure{ЦЕЛЬ РАБОТЫ}

Изучить постановку кооперативной игры и найти оптимальное распределение
выигрыша (дележ) между игроками путем вычисления компонент вектора Шепли.

\subsection*{Задание}

Для заданной характеристической функцией игры выполнить следующее:

\begin{itemize}
  \item Проверить кооперативную игру на супераддитивность и
        выпуклость. Если игра не супераддитивна, изменить характеристическую
        функцию таким образом, чтобы игра стала супераддитивной.
  \item Составить программу вычисления компонент вектора
        Шепли и рассчитать его.
  \item Проверить условия индивидуальной и групповой рационализации.
\end{itemize}

Вариант со значениями характерестической функции представлены в таблице~\ref{tab:tab01}.

\begin{table}[h!]
\centering
\caption{Задание характеристической функции}
\begin{tabular}{|c|c|}
\hline
\textbf{Множество} & \textbf{Значение }$v(I)$ \\ \hline
$\emptyset$ & 0 \\ \hline
$\{1\}$ & 2 \\ \hline
$\{2\}$ & 3 \\ \hline
$\{3\}$ & 4 \\ \hline
$\{4\}$ & 1 \\ \hline
$\{1,2\}$ & 6 \\ \hline
$\{1,3\}$ & 7 \\ \hline
$\{1,4\}$ & 4 \\ \hline
$\{2,3\}$ & 7 \\ \hline
$\{2,4\}$ & 5 \\ \hline
$\{3,4\}$ & 6 \\ \hline
$\{1,2,3\}$ & 12 \\ \hline
$\{1,2,4\}$ & 9 \\ \hline
$\{1,3,4\}$ & 9 \\ \hline
$\{2,3,4\}$ & 10 \\ \hline
$\{1,2,3,4\}$ & 14 \\ \hline
\end{tabular}
\label{tab:tab01}
\end{table}
