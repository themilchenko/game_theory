\conclusion

В данной работе было исследовано решение антагонистической игры двух лиц с нулевой суммой 
с использованием аналитического (метода обратной матрицы) и численного (метода Брауна--Робинсон) 
подходов.

\begin{align*}
v &= 12{,}992 \\
x^* &= (0{,}530,\ 0{,}114,\ 0{,}356) \\
y^* &= (0{,}341,\ 0{,}129,\ 0{,}530)
\end{align*}

Недостатки метода включают требования к квадратности и невырожденности матрицы игры, 
а также кубическую вычислительную сложность $O(n^3)$, где $n$~-- число стратегий.

Численный метод Брауна--Робинсон при $\varepsilon \leq 0{,}1$ дал приближённое решение:
\begin{align*}
v &= 13{,}009 \\
x^{*} &= (0{,}558,\ 0{,}118,\ 0{,}324) \\
y^{*} &= (0{,}324,\ 0{,}148,\ 0{,}528)
\end{align*}
Метод имеет линейную сложность $O(m + n)$ и свободен от ограничений аналитического метода, 
но обладает немонотонностью сходимости.

Сравнительный анализ показал, что метод Брауна--Робинсон обеспечивает решение 
в пределах заданной точности $\varepsilon$, что подтверждает его пригодность для 
приближённого решения произвольных матричных игр.

Таким образом, метод Брауна--Робинсон представляет собой эффективный инструмент 
для решения игр большой размерности.
