\structure{ЦЕЛЬ РАБОТЫ}

Изучить метод обратной индукции и его при- менение к решению конечных позиционных игр с полной
информацией. Изучить свойства решений таких игр.

\subsection*{Задание}

Найти решение конечношаговой позиционной игры с полной информацией. Для этого сгенерировать и
построить дерево случайной игры согласно варианту, используя метод обратной индукции, найти
решение игры и путь (все пути, если он не единственный) к этому решению. Обозначить их на дереве.

Данные о генерации дерева по заданному варианту представлены в таблице~\ref{tab:tab01}.

\begin{table}[h]
\centering
\caption{Значения по варианту}
\begin{tabularx}{\textwidth}{|X|X|X|X|X|} \hline
Номер варианта & Глубина дерева & Количество игроков & Количество стратегий & Диапазон выигрышей \\ \hline
12 & 7 & 2 & 2, 3 & \([-5,\ 25]\) \\ \hline
\end{tabularx}
\label{tab:tab01}
\end{table}
