\conclusion

В данной работе был исследован метод обратной индукции на примере задач, представленных в
виде двух деревьев решений с различной глубиной. Вначале рассмотрено дерево по варианту
глубиной 7, что позволило детально проанализировать процесс принятия решений в отдельных
ключевых узлах при наличии большого количества вариантов и высокой структурной сложности.
Такой поэтапный разбор продемонстрировал, как, начиная с конечных результатов, можно
последовательно «поднимать» оптимальные выигрыши на более высокие уровни дерева,
обеспечивая выбор оптимальной стратегии на каждом этапе.

Далее, для полноты картины, было взято дерево с глубиной 4, чтобы отобразить работу
метода на одной картинке сразу, где будут показаны выигрышные стратегии на каждом шаге
одновременно.
