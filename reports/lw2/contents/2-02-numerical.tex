\section{Численный метод}

Для численного решения игры с непрерывной функцией выигрыша используется метод
дискретизации ядра на равномерной сетке. На каждой итерации увеличивается шаг сетки, что
позволяет уточнять приближённое значение седловой точки.

Введём параметр разбиения \(N\) и для всех \(N = 1, 2, \ldots\) зададим аппроксимацию
функции ядра на единичном квадрате. Для этого разобьём отрезок \([0, 1]\) на \(N\)
равных частей с шагом \(\Delta = \frac{1}{N}\). Тогда узлы сетки будут иметь координаты:
\[
x_i = \frac{i}{N}, \quad y_j = \frac{j}{N}, \quad i, j = 0, \ldots, N.
\]

На этой сетке формируется матрица приближённых значений функции выигрыша:
\[
H^{(N)} = \left(H_{ij}^{(N)}\right), \quad \text{где} \quad H_{ij}^{(N)} = H\left( \frac{i}{N}, \frac{j}{N} \right), \quad i, j = 0, \ldots, N.
\]

Если на шаге с номером \(N\) седловая точка не обнаружена, применяется метод
Брауна–Робинсон для численного поиска решения соответствующей матричной игры.

Алгоритм завершает работу, когда разность между оценками цены игры на последних \(k\)
итерациях (по умолчанию \(k = 5\)) становится меньше заданного порога точности
\(\varepsilon = 0.01\). При увеличении количества итераций рекомендуется также увеличивать
параметр \(k\), чтобы обеспечить стабильность сходимости.

Вывод программы алгоритма для первых 8 итераций представлен на рисунке~\ref{fig:fig2}.

\begin{center}
\label{fig:fig2}
\begin{codelisting}
N = 2
[    0  -12.7  -18.7 ]
[-10.5  -13.2  -9.17 ]
[  -26  -18.7  -4.67 ]
Brown-Robinson solution:
x = 0.500 y = 0.500 H = -13.167

N = 3
[    0  -9.19  -15.4  -18.7 ]
[-6.44  -11.2   -13   -11.8 ]
[-15.1  -15.4  -12.7  -7.11 ]
[  -26  -21.9  -14.7  -4.67 ]
Brown-Robinson solution:
x = 0.333 y = 0.667 H = -12.963

N = 4
[    0  -7.17  -12.7  -16.5  -18.7 ]
[-4.62  -9.29  -12.3  -13.6  -13.3 ]
[-10.5  -12.7  -13.2   -12   -9.17 ]
[-17.6  -17.3  -15.3  -11.6  -6.29 ]
[  -26  -23.2  -18.7  -12.5  -4.67 ]
Brown-Robinson solution:
x = 0.500 y = 0.500 H = -13.167

N = 5
[    0  -5.87  -10.7  -14.4  -17.1  -18.7 ]
[ -3.6  -7.87  -11.1  -13.2  -14.3  -14.3 ]
[   -8  -10.7  -12.3  -12.8  -12.3  -10.7 ]
[-13.2  -14.3  -14.3  -13.2  -11.1  -7.87 ]
[-19.2  -18.7  -17.1  -14.4  -10.7  -5.87 ]
[  -26  -23.9  -20.7  -16.4  -11.1  -4.67 ]
Saddle point found:
x = 0.400 y = 0.600 H = -12.800

N = 6
[    0  -4.96  -9.19  -12.7  -15.4  -17.4  -18.7 ]
[-2.94  -6.80  -9.91  -12.3  -13.9  -14.8  -14.9 ]
[-6.44  -9.19  -11.2  -12.4   -13   -12.7  -11.8 ]
[-10.5  -12.1   -13   -13.2  -12.6  -11.2  -9.17 ]
[-15.1  -15.6  -15.4  -14.4  -12.7  -10.3  -7.11 ]
[-20.3  -19.7  -18.4  -16.3  -13.5  -9.91  -5.61 ]
[  -26  -24.3  -21.9  -18.7  -14.7  -10.1  -4.67 ]
Brown-Robinson solution:
x = 0.333 y = 0.667 H = -12.963

N = 7
[    0  -4.30  -8.05  -11.3  -13.9  -16.1  -17.6  -18.7 ]
[-2.49  -5.97  -8.91  -11.3  -13.2  -14.5  -15.2  -15.4 ]
[-5.39  -8.05  -10.2  -11.8  -12.8  -13.3  -13.2  -12.6 ]
[-8.69  -10.5  -11.9  -12.6  -12.8  -12.5  -11.6  -10.2 ]
[-12.4  -13.4  -13.9  -13.9  -13.3  -12.1  -10.4  -8.22 ]
[-16.5  -16.7  -16.4  -15.6  -14.1  -12.2  -9.67  -6.63 ]
[-21.1  -20.5  -19.3  -17.6  -15.4  -12.6  -9.31  -5.44 ]
[  -26  -24.6  -22.6  -20.1  -17.1  -13.5  -9.35  -4.67 ]
Brown-Robinson solution:
x = 0.429 y = 0.571 H = -12.830

N = 8
[    0  -3.79  -7.17  -10.1  -12.7  -14.8  -16.5  -17.8  -18.7 ]
[-2.16  -5.32  -8.07  -10.4  -12.3  -13.8  -14.9  -15.6  -15.8 ]
[-4.62  -7.17  -9.29   -11   -12.3  -13.2  -13.6  -13.7  -13.3 ]
[-7.41  -9.32  -10.8  -11.9  -12.6  -12.8  -12.7  -12.1  -11.1 ]
[-10.5  -11.8  -12.7  -13.1  -13.2  -12.8   -12   -10.8  -9.17 ]
[-13.9  -14.6  -14.8  -14.7  -14.1  -13.1  -11.7  -9.82  -7.57 ]
[-17.6  -17.7  -17.3  -16.5  -15.3  -13.7  -11.6  -9.17  -6.29 ]
[-21.7  -21.1  -20.1  -18.7  -16.8  -14.6  -11.9  -8.82  -5.32 ]
[  -26  -24.8  -23.2  -21.1  -18.7  -15.8  -12.5  -8.79  -4.67 ]
Brown-Robinson solution:
x = 0.375 y = 0.625 H = -12.823
\end{codelisting}
\captionof{figure}{Результаты работы численного алгоритма при разных значениях \(N\)}
\end{center}

Были найдены оптимальные стратегии игроков зa 42 итерации:

\[
x = 0{,}59, \quad y = 0{,}41,
\]
при которых значение функции выигрыша (цена игры) составляет:
\[
H(x^*, y^*) = -12{,}8.
\]
