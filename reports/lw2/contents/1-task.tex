\structure{ЦЕЛЬ РАБОТЫ}

Найти оптимальные стратегии непрерывной выпукло-вогнутой антагонистической игры
аналитическим и численным методами.

\subsection*{Задание}

Пусть функция выигрыша (ядро) антагонистической игры, заданной на единичном квадрате,
непрерывна:

\[
H(x,y) \in C(P), \quad P = [0,1] \times [0,1].
\]

Тогда существуют нижняя и верхняя цены игры, и, кроме того,
\[
\overline{h} = \max_{x} \min_{y} H(x,y), 
\quad
\underline{h} = \min_{y} \max_{x} H(x,y).
\]

Для среднего выигрыша игры имеют место равенства:
\[
\overline{h} = \int_{P} H(x,y)\, dF(x)\,dG(y), 
\quad
\underline{h} = \int_{P} H(x,y)\, dF(x)\,dG(y),
\]
где \(F(x), G(y)\) --- произвольные вероятностные меры распределения для обоих игроков,
заданные на единичном интервале.

Выпукло-ковогнутая игра всегда разрешима в смешанных стратегиях.

Пусть функция выигрыша игры имеет вид:
\[
H(x,y) = ax^2 + by^2 + cx + dy + e.
\]

Исходные данные для выполнения лабораторной работы приведены в таблице~\ref{tab:tab1}.

\begin{table}[h]
  \centering
  \caption{Вариант задания}
  \begin{tabularx}{\textwidth}{|X|X|X|X|X|X|}
    \hline
    \textbf{Номер варианта} & \textbf{a} & \textbf{b} & \textbf{c} & \textbf{d} & \textbf{e} \\ \hline
    12 & -10 & \(\frac{40}{3}\) & 40 & -16 & -32 \\ \hline
  \end{tabularx}
  \label{tab:tab1}
\end{table}
