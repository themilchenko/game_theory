\conclusion

В данной работе была исследована непрерывная антагонистическая выпукло-ковогнутая игра
двух лиц, а также аналитический и численный методы её решения.

Сначала было получено точное эталонное решение с помощью аналитического метода:
\[
x = 0{,}6;\quad y = 0{,}4;\quad H = -12{,}8.
\]

Затем был применён численный метод на основе аппроксимации функции выигрыша и
использования метода Брауна–Робинсон. При заданной точности \(\varepsilon \leq 0{,}01\)
было получено приближённое решение:
\[
x \approx 0{,}59;\quad y \approx 0{,}41;\quad H \approx -12{,}8.
\]

В заключение была проведена сравнительная оценка погрешностей. Полученные численные
значения хорошо согласуются с аналитическими, а отклонения по координатам \(x\) и \(y\)
не превышают допустимый уровень. Таким образом, численный метод показал высокую
эффективность и применимость для приближённого решения непрерывных выпукло-ковогнутых
игр.
