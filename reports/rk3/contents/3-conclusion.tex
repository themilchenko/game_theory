\conclusion

В ходе выполнения работы для 10 агентов была случайным образом сгенерирована стохастическая матрица доверия,
на основе которой получена результирующая матрица. Затем, используя выборку из случайных агентов, были заданы
функции выигрыша, определены целевые функции, найдена точка утопии и получено аналитическое решение игры с неполной
противоположностью интересов двух игроков. Для нахождения параметров uu и vv было выполнено дифференцирование целевых
функций и решена соответствующая система уравнений.

По сгенерированной матрице доверия был получен результат $X = 0.48$, $\Delta_{x_f} = 0.52$, $\Delta_{x_s} = 0.02$.
Так как $\Delta_{x_f} > \Delta{x_s}$, победу одержал второй игрок.
