\conclusion

В результате выполнения данной лабораторной работы были изучены и проанализированы
критерии выбора оптимальных стратегий в неантагонистических бескоалиционных играх.
Исследование позволило:

\begin{itemize}
    \item Детально рассмотреть понятия равновесия по Нэшу и оптимальности по Парето,
          выделив существенные особенности их определения и интерпретации в контексте
          игр нескольких игроков.
    \item Разработать алгоритмы для поиска ситуаций равновесия по Нэшу и оптимальных по
          Парето решений в биматричных играх, что подтвердилось на примере случайно
          сгенерированных игр размером 10×10.
    \item Провести тестирование алгоритмов на известных примерах игр («Дилемма
          заключенного», «Семейный спор», «Перекресток»), что показало соответствие
          полученных результатов методическим рекомендациям и теоретическим выкладкам.
    \item Рассмотреть биматричную игру 2×2, заданную по варианту, и выявить, что наличие
          нескольких равновесных ситуаций в чистых стратегиях приводит к существованию
          дополнительного вполне смешанного равновесия в расширении игры, которое можно
          вычислить по аналитическим формулам.
\end{itemize}
