\structure{ЦЕЛЬ РАБОТЫ}

Изучить критерии выбора стратегий в неан- тагонистической бескоалиционной игре двух
игроков на основе равновесия Нэша и оптимальности по Парето. Проверить дан-
ные критерии на примере рассмотренных выше игр. Исследовать свойства оптимальных
решений неантагонистических бескоалиционных игр на примере биматричных (2×2)-игр.

\subsection*{Задание}

\begin{enumerate}
    \item Сгенерировать случайную биматричную игру (10×10). Найти ситуации, равновесные
    по Нэшу и оптимальные по Парето, а также пересечение множеств этих ситуаций. Выполнить
    проверку реализованных алгоритмов на примере трех известных
    игр: «Семейный спор», «Перекресток», «Дилемма заключенного».
    \item Для заданной биматричной (2×2)-игры Г(А, В), пользуясь теоремами о свойствах
    оптимальных решений, найти ситуации, равновесные по Нэшу, для исходной игры и для ее смешанного
    расширения.
\end{enumerate}

Исходная матрица, заданная по варианту:

\[
\begin{pmatrix}
(10, 7) & (0, 4) \\
(2, 1) & (9, 3)
\end{pmatrix}
\]
