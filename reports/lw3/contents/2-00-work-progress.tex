\structure{ХОД РАБОТЫ}

Для решения данной лабораторной работы был использован язык программирования Go.
В проекте был написан класс для решения бескоалиционных игр.

Пример запуска программы:

\begin{codelisting}[language=Bash]
    go run cmd/lw3/main.go
\end{codelisting}

\section*{Равновесие по Нешу}

Ситуацией равновесия по Нэшу называется линия поведения игроков, если она 
устойчива относительно их индивидуального отклонения, т.\,е. в этой 
ситуации ни одному игроку не выгодно изменять своё мнение о выбранной 
стратегии при сохранении линии поведения другими игроками, так как он 
первым же от этого и пострадает.

Пусть $x^* = \bigl(x_1^*,\, x_2^*,\, \dots,\, x_n^*\bigr)$ --- 
ситуация (набор стратегий) всех игроков. Говорят, что $x^*$ является 
\emph{ситуацией равновесия по Нэшу} (в чистых стратегиях), если 
для каждого игрока $i = 1,2,\dots,n$ и любой стратегии $x_i \in X_i$ 
справедливо неравенство
\[
    H_i\bigl(x_i^*, x_{-i}^*\bigr) \; \ge \;
    H_i\bigl(x_i, x_{-i}^*\bigr),
\]
где $x_{-i}^*$ обозначает стратегии всех игроков, кроме игрока~$i$, 
а $H_i$ --- функция выигрыша (или полезности) игрока~$i$.

Совокупность всех равновесных по Нэшу ситуаций игры называется 
\emph{множеством равновесий Нэша}.

\section*{Оптимальность по Парето}

Ситуацией, оптимальной по Парето, называется состояние системы, 
при котором значение каждого частного критерия, описывающего 
состояние системы, не может быть улучшено без ухудшения положения 
других элементов. 

Рассмотрим множество векторов 
\[
  \{H(x)\} = \{H_1(x), H_2(x), \dots, H_n(x)\}, 
  \quad x \in X, 
  \quad x = (x_1, x_2, \dots, x_n),
\]
образованных значениями вектор-выигрышей игроков во всех возможных 
ситуациях $x \in X$. 

Ситуация $x^*$ в бескоалиционной игре $\Gamma$ называется 
\emph{оптимальной по Парето}, если не существует ситуации 
$x \in X$, для которой выполняется неравенство
\[
  H_i(x) \; \ge \; H_i(x^*)
\]
хотя бы для одного $i_0 \in N$, и при этом хотя бы для одного 
игрока $j \in N$ строгое неравенство 
\[
  H_j(x) \; > \; H_j(x^*).
\]
